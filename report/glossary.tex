\chapter{Glossary}\label{ch:glossary}
\paragraph{3DES} Triple DES (Data Encryption Standard). \textit{A symmetric key encryption algorithm (that is considered to be broken).}
\paragraph{AES} Advanced Encryption Standard. \textit{A symmetric key encryption algorithm.}
\paragraph{BAC} Basic Access Control. \textit{A protocol for authentication and key establishment using symmetric crypto challenge response (3DES), based on a key derived from the MRZ (document number, date of birth and date of expiry). Required to communicate with the an e-passport.}
\paragraph{CA (1)} Certificate Authority. \textit{A trusted third party that issues certificates.}
\paragraph{CA (2)} Chip Authentication. \textit{Authentication protocol of an eMRTD's for authenticating possession of the CA private key of a chip and for setting up stronger keys for further communication.}
\paragraph{CAN} Chip Authentication Number. \textit{A number that is used to authenticate a terminal towards an e-passport. It is a simple number printed om a card (like a password).}
\paragraph{CRL} Certificate Revocation List. \textit{A list of revoked certificates.}
\paragraph{CSCA} Country Signing Certificate Authority. \textit{The organization that is responsible for a country’s PKI.}
\paragraph{DG1} Data Group 1. \textit{A data group of an eMRTD that contains the personal data of the document holder.}
\paragraph{DG2} Data Group 2. \textit{A data group of an eMRTD that contains the facial image of the document holder.}
\paragraph{DG14} Data Group 2. \textit{A data group of an eMRTD that contains the Chip Authentication public key.}
\paragraph{DH} Diffie-Hellman. \textit{A key agreement protocol.}
\paragraph{DSC} Document Signer Certificate. \textit{A certificate on the EMRTD from the document signer.}
\paragraph{DS} Document Signer. \textit{The party that signs documents. The DS certificate is signed by the CSCA.}
\paragraph{ECDH} Elliptic Curve Diffie-Hellman. \textit{A key agreement protocol based on elliptic curve cryptography.}
\paragraph{EFsod} Document Security Document. \textit{A file on an eMRTD that contains hashes on all data groups, and a signature and public key from the Document Signer.}
\paragraph{E-passport} Electronic Passport. \textit{See eMRTD.}
\paragraph{eID} Electronic Identity Document. \textit{See eMRTD.}
\paragraph{eMRTD} Electronic Machine Readable Travel Document. \textit{See MRTD. A machine-readable travel document that contains an integrated circuit chip (IC) for contactless communication. Recognizable by a specific ICAO logo.}
\paragraph{HSM} Hardware Security Module. \textit{A device that is used to store cryptographic keys.}
\paragraph{ICAO} International Civil Aviation Organization. \textit{An agency of the United Nations that is responsible for the safety and security of international air travel. It standardizes machine-readable travel documents.}
\paragraph{IC} Integrated Circuit. \textit{A microchip that contains a memory and a processor and allows for communication with a different system.}
\paragraph{MRTD} Machine Readable Travel Document. \textit{A passport or other travel document that contains a machine-readable zone (MRZ) and that is designed to be read by a machine.}
\paragraph{MRZ} Machine Readable Zone. \textit{The area of a machine-readable travel document that is designed to be read by a machine. It contains, among other things, the name of the document holder, the date of birth, the document number and the date of expiry.}
\paragraph{NFC} Near Field Communication. \textit{A technology that allows for contactless communication between a reader and a tag.}
\paragraph{PACE} Password Authenticated Connection Establishment. \textit{A protocol for authentication and key establishment using asymmetric crypto challenge response. After key agreement, authentication can happen based on the MRZ or CAN. An more secure alternative to BAC.}
\paragraph{PA} Passive Authentication. \textit{The process of checking the hashes and digital signatures (from the SO\_D) on the data on the document.}
\paragraph{PCD} Proximity Coupling Device. \textit{The NFC terminology for the terminal that is connecting.}
\paragraph{PICC} Proximity Integrated Circuit Card. \textit{The NFC terminology for the card that is connecting.}
\paragraph{PKI} Public Key Infrastructure. \textit{}
\paragraph{RSA} Rivest–Shamir–Adleman. \textit{An asymmetric cryptosystem.}
