\chapter{Further research}
\label{ch:further-research}
In this chapter, we will discuss the further research that can be done on the topic of application of Remote Document Encryption.
These are more fundamental research questions, in contrast to the more implementation-related topics that were discussed in the previous chapter.

\section{Split-key architecture}
\label{sec:split-key-architecture}
Currently, the decryption (key retrieval) step in RDE is based fully on the e-passport.
The secret key is retrieved fully from the e-passport.
This has some consequences.
In case a user loses his e-passport, however, there is no way for them to revoke their passport.
They surely could indicate at a key server that their passport is lost (after authenticating with their email, for example), hopefully preventing others from still using the passport's enrollment parameters to encrypt for.
Already encrypted documents that are already out there, however, cannot be revoked.

A possible solution to this problem is to use a split-key architecture.
In such an architecture, the secret key would not only be based on the response of the e-passport, but also on a secret key that is stored on a key server.
For decryption, the key server would thus also need to be involved.
To revoke a document, the user would simply need to revoke the key on the key server, making any future decryption attempts fail.

With such an infrastructure, also a PIN could be introduced that needs to be provided in order to unlock the key server's secret key part.
Introduction of a PIN would give us even more security, as it prevents stolen or lost passports from being used to retrieve the secret message key.
Note that this still does not bind the e-passport to a specific reader app.
The reader app itself still does not need to store any secret data itself.

\section{Biometric authentication}
\label{sec:biometric-authentication}
A PIN would be a nice addition to the RDE scheme, but it is not the only way to improve the security of the scheme.
A PIN can still leak or be shared with others.
An interesting alternative is to use biometric authentication.

Currently, SURF's eduID project is working on a system that authenticates users with their passport and biometric data.
This is based on a product from the company Verifai\footnote{\url{https://www.verifai.com}}.
Verifai offers a so-called \emph{liveness check}\footnote{\url{https://docs.verifai.com/sdk/android/components/liveness-check}} to verify the user is indeed the document holder and is actually present.
This is done by asking the user to blink, move their head, speak some words, and verifying this against the facial image in the passport.

It could be interesting to use this liveness check to authenticate the document holder both at enrollment, but more importantly, at decryption.
Only after the liveness check is performed, the key server, implementing a split-key architecture, should release its secret key part.
A drawback of this approach, however, is that the Verifai liveness check is performed in-app, and thus requires a certified reader app (we need to trust the app to perform the liveness check correctly).
This is a big change from the current prototype, where the reader app is not certified.

\section{Other readers}
\label{sec:other-readers}
Currently, the prototype uses an Android app as reader for e-passports (and an iOS app is still to be developed).
This is chosen as NFC-capable phones are commonly available at this moment.
However, there is no reason why reading e-passports should be limited to phones.
For example, a simple USB smart card reader could be used to read e-passports as well.

It would be interesting to investigate if it is possible to interface with a USB smart card reader from a web browser, and if so, how this could be used to read e-passports.
This way, using a reader app would not be required anymore.
This would also further simplify the infrastructure, because no proxy server and decryption handshake between browser and reader app would be required anymore.




